%\documentclass[a4paper,landscape,11pt]{article}
\documentclass[a4paper,11pt]{article} 
%\usepackage{pdflscape}
\usepackage[paper=portrait,pagesize]{typearea}
%\usepackage[T1,T2A]{fontenc}
\usepackage[utf8x]{inputenc}
%\usepackage[english,ukrainian,russian]{babel} 
\usepackage[english,russian]{babel} 
\usepackage{wrapfig}
\usepackage[table,xcdraw]{xcolor}
\usepackage{booktabs}
\usepackage{pifont}
\usepackage{graphicx}
\graphicspath{ {images/} }

\usepackage{tikz}
\usepackage{siunitx}
\usepackage[american,cuteinductors,smartlabels]{circuitikz}

\usepackage{hyperref}

\usepackage{advdate}
%\usepackage{showframe} % для отладки позиции на странице
\usepackage{cancel}

%\setlength{\voffset}{-72pt} %отступ сверху - чтобы увидеть откомментарить \usepackage{showframe}
%\setlength{\voffset}{-36pt} %landscape
%\setlength{\footskip}{-20pt} %landscape
\setlength{\topmargin}{0pt} 
%\setlength{\headheight{1pt}
\setlength{\headsep}{0pt}
%\setlength{\hoffset}{-72pt} %landscape
\setlength{\marginparwidth}{0pt}
%\setlength{\textwidth}{530pt} %landscape
%\setlength{\textheight}{808pt} %landscape


%\author{ Прокшин Артем \\
%\small ЛЭТИ\\
%\small \texttt{taybola@gmail.com}}

%\date{}
%abcdefghijklmnop


\newcommand*\OK{&\small \ding{51}$\!\!_\circ$} % начал защищать
\newcommand*\Ok{&\small \ding{51}$\!\!_\circ$} % начал защищатi
\newcommand*\ok{&{\small \ding{51}}} % присутствовал
\newcommand*\oK{&{\small \ding{51}?}} % присутствовал?
\newcommand*\no{&{\small }} % отсутствовал
\newcommand*\D{\tiny\ding{48}} % защита, defend
\newcommand*\da{&{\small\ding{48}$\!\!_1$}} % защита, defend
\newcommand*\dab{&{\small\ding{48}$\!\!^1_2$}} % защита, defend
\newcommand*\ab{&{\small\ding{48}$\!\!^1_2$}} % защита, defend
\newcommand*\ad{&{\small${}^1\!\!$\ding{48}$\!\!_4$}} % защита, defend
%\newcommand*\ab{&{\small\ding{48}$\!\!^1_2$}} % защита, defend
\newcommand*\bc{&{\small\ding{48}$\!\!^2_3$}} % защита, defend
\newcommand*\dabc{&{\small\ding{48}$\!\!^1_{23}$}} % защита, defend
\newcommand*\dabcd{&{\small\ding{48}$\!\!^{12}_{34}$}} % защита, defend
\newcommand*\ac{&{\small\ding{48}$\!\!^1_{23}$}} % защита, defend
\newcommand*\db{&{\small\ding{48}$\!\!_2$}} % защита, defend
\newcommand*\dc{&{\small\ding{48}$\!\!_3$}} % защита, defend
\newcommand*\dd{&{\small\ding{48}$\!\!_4$}} % защита, defend
\newcommand*\bd{&{\small${}^2\!\!$\ding{48}$\!\!^3_{4}$}} % защита, defend
\newcommand*\de{&{\small\ding{48}$\!\!_5$}} % защита, defend
\newcommand*\dE{&{\small${}^4\!\!\!$\ding{48}$\!\!_5$}} % защита, defend
\newcommand*\cd{&{\small\ding{48}$\!\!^3_4$}} % защита, defend
\newcommand*\dg{&{\small\ding{48}$\!\!_6$}} % защита, defend
\newcommand*\fg{&{\small${}^6\!\!$\ding{48}$\!\!_7$}} % защита, defend
\newcommand*\dH{&{\small\ding{48}$\!\!_8$}} % защита, defend
\newcommand*\gh{&{\small\ding{48}$\!\!^7_8$}} % защита, defend
\newcommand*\fh{&{\small\ding{48}$\!\!^7_{89}$}} % защита, defend 
\newcommand*\ce{&{\small${}^3\!\!$\ding{48}$\!\!_5$}} % защита, defend
\newcommand*\ef{&{\small${}^5\!\!$\ding{48}$\!\!_6$}} % защита, defend
%\newcommand*\dh{&{\small\ding{48}$\!\!_8$}} % защита, defend
\newcommand*\di{&{\small\ding{48}$\!\!_9$}} % защита, defend
\newcommand*\cdef{&{\small ${}^2_4\!\!$\ding{48}$\!\!^{3}_{5}$}} % защита, defend
\newcommand*\cde{&{\small ${}^2\!\!$\ding{48}$\!\!^{3}_{5}$}} % защита, defend
\newcommand*\efg{&{\small ${}^5\!\!$\ding{48}$\!\!^{6}_{7}$}} % защита, defend
\newcommand*\befgh{&{\small ${}_2^5\!\!$\ding{48}$\!\!^{6}_{78}$}} % защита, defend
\newcommand*\Dh{&{\small${}^4\!\!$\ding{48}$\!\!_8$}} % защита, defend
\newcommand*\cfg{&{\small ${}^3\!\!$\ding{48}$\!\!^{6}_{7}$}} % защита, defend
\newcommand*\fgh{&{\small ${}^6\!\!$\ding{48}$\!\!^{7}_{8}$}} % защита, defend
\newcommand*\bce{&{\small ${}^2\!\!$\ding{48}$\!\!^{3}_{5}$}} % защита, defend
\newcommand*\dO{&{\small\ding{48}$\!\!_{15}$}}
\newcommand*\Skip{\noindent\rule{0.3cm}{0.9pt}}


\begin{document}
%\thispagestyle{empty}
% or
\pagenumbering{gobble}
%\AdvanceDate[-1] % печатаю в субботу а нужна пятница
\begin{center}\today\end{center}
\vspace*{1\baselineskip} %landscape

%\begin{table} \centering 
%\hspace{-6cm} % landscape
\hspace{-2cm} % portrait
\newcommand*{\CS}{9pt} % ширина колонки
\begin{tabular}{p{7pt}|l|p{\CS}|p{\CS}|p{\CS}|p{\CS}|p{\CS}|p{\CS}|p{\CS}|p{\CS}|p{\CS}}
%\multicolumn{16}{c}{График выполнения лабораторных работ студентами 8871 группы} \\ 
\multicolumn{11}{c}{Ведомость посещения занятий по датчикам студентами 7492 группы} \\
\toprule 
&&&&&&&&&&\\
&&&&&&&&&&\\
&&&&&&&&&&\\
&&&&&&&&&&\\
&&&&&&&&&&\\
&&&&&&&&&&\\
&&\rotatebox{90}{\rlap{\small 5 марта ( ОУ )}}
&\rotatebox{90}{\rlap{\small 19 марта (инстр.У)}} 
&\rotatebox{90}{\rlap{\small 2 апреля }}
&\rotatebox{90}{\rlap{\small 16 апреля }}
&\rotatebox{90}{\rlap{\small 30 апреля}}
&\rotatebox{90}{\rlap{\small 14 мая }}
&\rotatebox{90}{\rlap{\small 28 мая }}
&\rotatebox{90}{\rlap{\small }}
\\
\midrule 
1\,&Александрова Анастасия Петровна \ok\ok\ok\ok\ok\ok\ok&&\\
2\,&Белик Владислав                 &б \ok\ok\ok\ok\no\ok&&\\
3\,&Беловенцова Алина Альбертовна   \ok\ok\ok\ok\ok\ok\ok&&\\
4\,&Беляев Владимир Алексеевич      \no\no\no\no\no\no\no&&\\
5\,&Бондарчук Александр Павлович    \ok\ok\ok\ok\ok\ok\ok&&\\
\midrule
6\,&Бударина Марина                 \ok\ok\ok\ok\ok\ok\ok&&\\
7\,&Гоган Владислав Викторович      \ok\no\ok\ok\ok\ok\ok&&\\
8\,&Дружинин Антон Артемович        \ok\ok\ok\ok\ok\ok\ok&&\\
9\,&Исаков Антон Андреевич          \ok\ok\ok\ok\no\ok\ok&&\\ 
10\,&Комаров Денис Михайлович(ст.)  \ok\ok\no\no\ok\ok\ok&&\\
\midrule
11\,&Кушнерев Михаил Игоревич       \ok\ok\ok\ok\ok\ok\no&&\\
12\,&Лисицын Егор Николаевич        \no\no\ok\no\no\no\no&&\\
13\,&Малашевская Екатерина          \ok\ok\ok\ok\ok\no\ok&&\\
14\,&Малинина Анастасия Дмитриевна  \ok\ok\ok\ok\ok\ok\ok&&\\
15\,&Мигранов Руслан Михайлович     \ok\ok\ok\ok\ok\ok\ok&&\\
\midrule
16\,&Мыдлык Виталий Андреевич       \ok\ok\ok\ok\ok\no\no&&\\
17\,&Павлов Максим Андреевич        \ok\ok\ok\ok\ok\ok\ok&&\\
18\,&Раудонис Ян Вацславович        \ok\ok\ok\ok\ok\ok\ok&\\ 
19\,&Репин Павел Олегович           \ok\ok\ok\ok\ok\ok\ok&\\
20\,&Селезнев Владимир Алексеевич   \ok\ok\ok\ok&б &б &&\\
\midrule
21\,&Строгалев Павел Николаевич     \ok\ok\ok\ok\ok\ok\ok&\\
22\,&Шевченко Анастасия             &б \ok\ok\ok\ok\ok\ok&\\
\bottomrule
\end{tabular} 

\newpage
\KOMAoptions{paper=landscape,pagesize}
\recalctypearea

%
\hspace{-6.1cm} %landscape
\begin{tabular}{l|llccccccccccccc}
\multicolumn{10}{c}{выполнение лабораторнах работ, 7492 группа} \\
\toprule
&&Л1&Л1& Л2&Л2& Л3&Л3& Л4&Л4& Л5&Л5& Л6&Л6& \\
1\,&Александрова Анастасия Петровна &19.03&19.03 &30.03&30.03 &31.03&31.03&30.04& 8.05&10.05&10.05&21.05&21.05\\
2\,&Белик Владислав                 &19.03&19.03 & 2.04&26.05&16.04&16.04& 1.05&26.05&26.05&26.05\\
3\,&Беловенцова Алина Альбертовна   &19.03&19.03 &2.04&2.04  &16.04&16.04&28.05&28.05&15.05&26.05\\
4\,&Беляев Владимир Алексеевич      &&&&&&&&\\
5\,&Бондарчук Александр Павлович    &19.02&19.02 &2.04& 1.05  & 8.04& 8.04&29.04&29.04&16.05&26.05&29.05&29.05\\
\midrule
6\,&Бударина Марина                 &19.04&19.04 &22.04&22.04 &16.04&16.04& 1.05& 9.05&15.05&26.05\\
7\,&Гоган Владислав Викторович      &19.04&19.04 &2.04&2.04   &16.04&16.04&28.05& --- &28.05& ---\\
8\,&Дружинин Антон Артемович        &19.03&19.03 &2.04&9.05   &16.04&16.04& 1.05& 9.05&14.05&28.05&28.05&28.05\\
9\,&Исаков Антон Андреевич          &19.03&19.03 &2.04&2.04   &16.04&24.05&29.04& 9.05&15.05&24.05\\
10\,&Комаров Денис Михайлович       &19.03&19.03 &    &       &     &     &30.04& --- &15.05&15.05&29.05&29.05\\
\midrule
11\,&Кушнерев Михаил Игоревич       &19.03&19.03 &30.03& 7.05 &16.04&16.04& 1.05&26.05&15.05&26.05&26.05&27.05\\
12\,&Лисицын Егор Николаевич        &&&&&&&&\\
13\,&Малашевская Екатерина          &19.03&19.03 &30.03&30.03 &16.04&16.04&30.04&22.05&27.05&28.05\\
14\,&Малинина Анастасия Дмитриевна  &19.03&19.03 &2.04&2.04   &17.04&17.04& 5.05& 7.05&27.05&28.05\\
15\,&Мигранов Руслан Михайлович     &19.03&19.03 &30.03&30.03 &31.03&31.03&30.04& 8.05&10.05&10.05&23.05&23.05\\
\midrule
16\,&Мыдлык Виталий Андреевич       &19.03&19.03 &2.04&2.04   &15.04&15.04&30.04& --- & 1.05& 7.05\\
17\,&Павлов Максим Андреевич        &19.03&19.03 &3.04&9.05   &17.04&17.04&15.05&24.05&15.05&24.05&28.05& ---\\
18\,&Раудонис Ян Вацславович        &19.03&19.03 &2.04&2.04   &16.04&16.04&15.05&24.05&15.05&24.05\\
19\,&Репин Павел Олегович           &19.03&19.03 &3.04&3.04   &17.04&17.04&30.04& 9.05&14.05&24.05\\
20\,&Селезнев Владимир Алексеевич   &19.03&19.03 &30.30&--    &16.04&16.04&&\\
\midrule
21\,&Строгалев Павел Николаевич     &19.03&19.03 &2.04&13.04  &17.04&17.04&29.04&29.04&15.05&26.05\\
22\,&Шевченко Анастасия             &19.03&19.03 &2.04&2.04   &16.04&16.04&29.04&29.04&28.05&28.05&30.05&30.05\\
\bottomrule
\end{tabular}

\newpage
\KOMAoptions{paper=portrait,pagesize}
\recalctypearea
\section*{лаб6}
Александрова -- все есть, отчет в Компасе (3D v17.1)

Дружинин -- измерение температуры с нестабильным питанием от выпрямителя без фильтра будет давать неверный результат, 

Комаров  -- все есть, отчет в Компасе (3D v18.1) однако нет в свойствах pdf-файла автора, темы, клбчевых слов.
в документе от Компаса почему-то отсутствует надпись по краю, что это учебная версия программы.
Буду предполагать что копия программы лицензионная -- а поэтому В	ы имеете право спросить разработчиков Компаса
Каким образом вставить pdf/A теги в документ: автор, тема, дата, ключевые слова. 
И если такой возможности в настоящий момент нет, то запросить чтобы внесли изменения

Мигранов -- датчик температуры, сопротивление изменряется в мостовой схеме.

Павлов -- неясна причина использования выпрямителя напряжения на второй фазы VM3, в отчете ничего нет.

\section*{лаб5}

Белик -- добавлен формат pdf/A

Беловенцова -- сам отчет в порядке, в pdf/A ни одного поля

Бондарчук -- все есть, вывод формулы желательно поподробнее.

Бударина -- все есть, есть вывод формулы

Дружинин -- в pdf/A какой-то Томсон и не по-русски.

Комаров -- отчет в Компасе, нет кодов

Кушнерев -- все есть, pdf/A поля есть

Малашевская -- все есть,

Малинина -- все есть, в свойствах pdf-файла автор некто Acer

Мыдлык -- нет вывода формулы полосового фильтра, при добавлении гармоник не введена указанная в индивидуальном задании фаза

Раудонис -- все есть, вывод формул тоже!

Строгалев -- неясен график на рис 3. все поля pdf/A пустые

Шевченко -- все есть, вывод формулы желательно поподробнее.

\newpage
\subsection*{лаб4}

Александрова -- все есть, отчет в Компасе (3D v17.1) всегда выигрышнее по сравнению с вордом (дополнительные плюсы)

Белик --   Измеряемое напряжение на АЦП(рис 4) варьируется с 1.3 до 1.72в, что занимает крохотную часть от возможностей АЦП 0..3.3В;
           расширен диапазон дла АЦП, но можно было бы и больше расширить, добавлен формат pdf/A

Бондарчук -- все есть

Бударина -- из рис.4 выходной сигнал является суммой 1.5 и 0.63, что входит в диапазон (0…3)В, это верно, но диапазон можно было бы еще увеличить,
             до $2/3\approx 1$в от 1.5в (здесь учтено, что напряжение может быть на 50\% выше номинального)

Дружинин -- все есть

Исакаев -- для входного напряжения приведено действующее значение

Комаров -- выполняет 4ю работу, но в теме, обозначенной в шаблоне ЕСКД стоит актиывные выпрямители. Измеряемое напряжение на АЦП(рис 4) варьируется с 0.95 до 1.12в, что
занимает крохотную часть от возможностей АЦП 0..3.3В


Кушнерев -- на микроконтролллер подается сигнал с малым размахом амплитуды=0.2в, размах амплитуды может достигать 1вольт  (стр.6
должно быть $\left|U_{max} - U_\textcyrillic{сдвига}\right|\cdot1.5 < 1.5$вольт, а студент ошибочно предположил что $\left|U_{max} - 0\right|\cdot1.5 < 1.5$вольт   
(15 мая исправил)

Малашевская -- почему напряжение на АЦП на рис 3 равно 2230В хотя в отчете читаем "Из рис.3 видно, что амплитуда сигнала =1В. Сигнал выхода 1,5·1+1,5В=3В
входит в интервал (0;3,3) В. (Здесь 1,5В- сигнал смещения)."

Мигранов -- все есть, отчет в Компасе -- плюс к оценке

Мыдлык -- нет графиков напряжения на входе микроконтроллера

Павлов -- все есть, нет в свойсвах файла индивидуальных pdf-свойств (ФИО, тема, дата, ключевые слова)

Раудонис -- все есть

Репин -- все есть, в свойствах файла имя автора Vladimir

Строгалев -- всё есть.

Шевченко -- нет фамилии в шаблоне ЕСКД, напряжение на АЦП занимает от 0.8в до 2.3в, т.е. $1.5\pm0.8$, если предпоожить что измеряемое напряжение на 150\% выше
номинала, то $\pm0.8\cdot110\% = 1.2$, это значение 1.2 можно быо бы довести до 1.5в
\newpage
\subsection*{лаб3}
Александрова -- отчет в Компасе (3D v17.1) всегда выигрышнее по сравнению с вордом (дополнительные плюсы)

Белик -- всё есть, ОУ -- реальный

Беловенцова  -- всё есть, ОУ -- идеальный

Бондарчук -- почти всё есть (нет графиков по эскпериментальным данным), ОУ -- идеальный

Бударина -- коды, тоько tina, ОУ -- идеальный  (в свойствах файла стоит Кайгородов, в следующий раз поменяйте)

Гоган  -- все есть

Исаков -- не проверяю ворд, экспортируйте в pdf

Кушнерев -- реальный ОУ, все есть. узнаю стиль  Зиганшиной (и в свойствах файла написано Зиганшина)

Малашевская -- ?? %все есть, основная надпись ЕСКД на первой и последующих страницах должны отичаться.

Малинина 

Мигранов -- очет в Компасе (плюс к оценке)

Мыдлык -- кратко, но все есть

Павлов -- все есть

Раудонис -- ??

Репин -- все есть, реальный ОУ

Селезнев -- ?? %все есть.

Строгалев -- все есть 

Шевченко -- все есть

\newpage
\subsection*{лаб2}

Белик -- неясно, как определил <<Отношение напряжения смещения усилителя в зависимости от переменных R1 и R2>>.
, и почему сделан вывод 
<<При u1 = u2 = 0 В и при U- выходное напряжение. увеличении элемента R1 напряжение смещения увеличивается, а при увеличении R2 снижается.>>
нет тегов pdf/A


Дружинин -- переделал (коэф подавления синфазного сигнала драматически мал.)

Кушнерев -- неверно определен коэф подавления синфазных помех (должен быть около 2197, если коэф усиления синфазного сигнала взял теоретический =1 ), затем исправил.

Малашевская -- нет АЧХ

Мыдлык -- неясно, как определил <<Отношение напряжения смещения усилителя в зависимости от переменных R1 и R2>>.
, и почему сделан вывод 
<<При u1 = u2 = 0 В и при U- выходное напряжение. увеличении элемента R1 напряжение смещения увеличивается, а при увеличении R2 снижается.>>(исправлено)

Павлов -- выбран неверный коэф усиления. у каждого был коэф усиления 1NN0, где NN - номер в списке по порядку, у вас номер 17, т.е. коэф усиления должен быть равен 1170

Строгалев -- всё есть.
\newpage
\subsection*{лаб1}
Александрова -- pdf из Компас-3В 18.1 не прочитались шрифты, формулы есть.

Белик        -- $10^{-6}$ это микро, мкВ, а не милли (мВ)

Беловенцова     LM318

Бондарчук  перепутаны позиционные номера у сопротивлений

Бударина, в схеме 4 лучше поставить $R_1=1$k a $R_3=7$k

Гоган, в отчете не стоит какой именно ОУ, судя по экспортированному файлу со схемой, только стандартный

Дружинин  LM318 - к сожалению не прислал файл с реальным ОУ (желательно $R_1$ должно быть равно $R_2$)

Исаков  LM318 (желательно $R_1$ должно быть равно $R_2$ и $R_3$ = $R_4$  для симметрии)

Кушнерев LM324

Малашевская LM324

Малинина

Мигранов LM324


Павлов , (файлы Дружинина и Павлова - близнецы, плюс за то что если одинаково увеличить резисторы коэф усиления не поменяются, но входной ток может быть другой, и наппяжение смещения)
(желательно $R_1$ должно быть равно $R_2$)

Раудонис, 

Репин, LM324

Селезнев, LM324

Строгалев LM318

Шевченко
\end{document}
