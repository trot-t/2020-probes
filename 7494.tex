%\documentclass[a4paper,landscape,11pt]{article}
\documentclass[a4paper,11pt]{article} 
%\usepackage[T1,T2A]{fontenc}
\usepackage[utf8x]{inputenc}
%\usepackage[english,ukrainian,russian]{babel} 
\usepackage[english,russian]{babel} 
\usepackage{wrapfig}
\usepackage[table,xcdraw]{xcolor}
\usepackage{booktabs}
\usepackage{pifont}
\usepackage{graphicx}
\graphicspath{ {images/} }

\usepackage{tikz}
\usepackage{siunitx}
\usepackage[american,cuteinductors,smartlabels]{circuitikz}

\usepackage{hyperref}

\usepackage{advdate}
%\usepackage{showframe} % для отладки позиции на странице
\usepackage{cancel}

%\setlength{\voffset}{-72pt} %отступ сверху - чтобы увидеть откомментарить \usepackage{showframe}
%\setlength{\voffset}{-36pt} %landscape
%\setlength{\footskip}{-20pt} %landscape
\setlength{\topmargin}{0pt} 
%\setlength{\headheight{1pt}
\setlength{\headsep}{0pt}
%\setlength{\hoffset}{-72pt} %landscape
\setlength{\marginparwidth}{0pt}
%\setlength{\textwidth}{530pt} %landscape
%\setlength{\textheight}{808pt} %landscape


%\author{ Прокшин Артем \\
%\small ЛЭТИ\\
%\small \texttt{taybola@gmail.com}}

%\date{}
%abcdefghijklmnop


\newcommand*\OK{&\small \ding{51}$\!\!_\circ$} % начал защищать
\newcommand*\Ok{&\small \ding{51}$\!\!_\circ$} % начал защищатi
\newcommand*\ok{&{\small \ding{51}}} % присутствовал
\newcommand*\oK{&{\small \ding{51}?}} % присутствовал?
\newcommand*\no{&{\small }} % отсутствовал
\newcommand*\D{\tiny\ding{48}} % защита, defend
\newcommand*\da{&{\small\ding{48}$\!\!_1$}} % защита, defend
\newcommand*\dab{&{\small\ding{48}$\!\!^1_2$}} % защита, defend
\newcommand*\ab{&{\small\ding{48}$\!\!^1_2$}} % защита, defend
\newcommand*\ad{&{\small${}^1\!\!$\ding{48}$\!\!_4$}} % защита, defend
%\newcommand*\ab{&{\small\ding{48}$\!\!^1_2$}} % защита, defend
\newcommand*\bc{&{\small\ding{48}$\!\!^2_3$}} % защита, defend
\newcommand*\dabc{&{\small\ding{48}$\!\!^1_{23}$}} % защита, defend
\newcommand*\dabcd{&{\small\ding{48}$\!\!^{12}_{34}$}} % защита, defend
\newcommand*\ac{&{\small\ding{48}$\!\!^1_{23}$}} % защита, defend
\newcommand*\db{&{\small\ding{48}$\!\!_2$}} % защита, defend
\newcommand*\dc{&{\small\ding{48}$\!\!_3$}} % защита, defend
\newcommand*\dd{&{\small\ding{48}$\!\!_4$}} % защита, defend
\newcommand*\bd{&{\small${}^2\!\!$\ding{48}$\!\!^3_{4}$}} % защита, defend
\newcommand*\de{&{\small\ding{48}$\!\!_5$}} % защита, defend
\newcommand*\dE{&{\small${}^4\!\!\!$\ding{48}$\!\!_5$}} % защита, defend
\newcommand*\cd{&{\small\ding{48}$\!\!^3_4$}} % защита, defend
\newcommand*\dg{&{\small\ding{48}$\!\!_6$}} % защита, defend
\newcommand*\fg{&{\small${}^6\!\!$\ding{48}$\!\!_7$}} % защита, defend
\newcommand*\dH{&{\small\ding{48}$\!\!_8$}} % защита, defend
\newcommand*\gh{&{\small\ding{48}$\!\!^7_8$}} % защита, defend
\newcommand*\fh{&{\small\ding{48}$\!\!^7_{89}$}} % защита, defend 
\newcommand*\ce{&{\small${}^3\!\!$\ding{48}$\!\!_5$}} % защита, defend
\newcommand*\ef{&{\small${}^5\!\!$\ding{48}$\!\!_6$}} % защита, defend
%\newcommand*\dh{&{\small\ding{48}$\!\!_8$}} % защита, defend
\newcommand*\di{&{\small\ding{48}$\!\!_9$}} % защита, defend
\newcommand*\cdef{&{\small ${}^2_4\!\!$\ding{48}$\!\!^{3}_{5}$}} % защита, defend
\newcommand*\cde{&{\small ${}^2\!\!$\ding{48}$\!\!^{3}_{5}$}} % защита, defend
\newcommand*\efg{&{\small ${}^5\!\!$\ding{48}$\!\!^{6}_{7}$}} % защита, defend
\newcommand*\befgh{&{\small ${}_2^5\!\!$\ding{48}$\!\!^{6}_{78}$}} % защита, defend
\newcommand*\Dh{&{\small${}^4\!\!$\ding{48}$\!\!_8$}} % защита, defend
\newcommand*\cfg{&{\small ${}^3\!\!$\ding{48}$\!\!^{6}_{7}$}} % защита, defend
\newcommand*\fgh{&{\small ${}^6\!\!$\ding{48}$\!\!^{7}_{8}$}} % защита, defend
\newcommand*\bce{&{\small ${}^2\!\!$\ding{48}$\!\!^{3}_{5}$}} % защита, defend
\newcommand*\dO{&{\small\ding{48}$\!\!_{15}$}}
\newcommand*\Skip{\noindent\rule{0.3cm}{0.9pt}}


\begin{document}
%\thispagestyle{empty}
% or
\pagenumbering{gobble}
%\AdvanceDate[-1] % печатаю в субботу а нужна пятница
\begin{center}\today\end{center}
\vspace*{1\baselineskip} %landscape

%\begin{table} \centering 
%\hspace{-6cm} % landscape
\hspace{-2cm} % portrait
\newcommand*{\CS}{9pt} % ширина колонки
\begin{tabular}{p{7pt}|l|p{\CS}|p{\CS}|p{\CS}|p{\CS}|p{\CS}|p{\CS}|p{\CS}|p{\CS}|p{\CS}}
%\multicolumn{16}{c}{График выполнения лабораторных работ студентами 8871 группы} \\ 
\multicolumn{11}{c}{Ведомость посещения занятий по датчикам студентами 7494 группы} \\
\toprule 
&&&&&&&&&&\\
&&&&&&&&&&\\
&&&&&&&&&&\\
&&&&&&&&&&\\
&&&&&&&&&&\\
&&&&&&&&&&\\
&&\rotatebox{90}{\rlap{\small 5 мартa ( ОУ )}}
&\rotatebox{90}{\rlap{\small 19 марта (инстр.У)}}
&\rotatebox{90}{\rlap{\small 2 апреля }}
&\rotatebox{90}{\rlap{\small 16 апреля }}
&\rotatebox{90}{\rlap{\small 30 апреля }}
&\rotatebox{90}{\rlap{\small 14 мая}}
&\rotatebox{90}{\rlap{\small 28 мая }}
&\rotatebox{90}{\rlap{\small }}
&\rotatebox{90}{\rlap{\small }}
\\ 
\midrule
1\,&Бибиков Владислав Вадимович        &\ok\ok\no\no\no&&\\
2\,&Вакуленко Алексей Владимирович     &\no\no\no\no\no&&\\
3\,&Грязнова Ольга Николаевна        \ok\no\no\no\no\no&&\\
4\,&Гуськов Илья Викторович            &\no\no\no\no\no&&\\
5\,&Заруба Роман Викторович            &\no\no\no\no\no&&\\
\midrule
6\,&Клейменов Кирилл                   &\no\no\no\no\ok&&\\
7\,&Куанышбек Канат                    &\no\no\no\no\no&&\\
8\,&Ломский Илья Александрович         &\no\no\no\no\no&&\\
9\,&Михайленко Владислав               &\no\no\no\no\no&&\\ 
10\,&Наумов Игорь Владимирович         &\no\no\no\no\no&&\\
\midrule
11\,&Нестерова Анастасия Глебовна      &\no\no\no\no\ok&&\\
12\,&Осокин Иван Сергеевич             &\no\no\no\no\no&&\\
13\,&Плахотников Олег Игоревич         &\no\no\no\no\no&&\\
14\,&Таран Семён Юрьевич               &\no\no\ok\ok\ok&&\\
15\,&Шейнкер Алексей                   &\no\no\no\no\ok&&\\
\bottomrule
\end{tabular} 

\newpage
%
%\hspace{-4.1cm} %landscape
\begin{tabular}{l|llccccccccccccc}
\multicolumn{10}{c}{выполнение лабораторнах работ, 7494 группа} \\
\toprule
&&Л1&Л1& Л2&Л2& Л3&Л3& Л4&Л4&пр.№7\\ 
\midrule
1\,&Бибиков Владислав Вадимович    &&&&&&&&\\
2\,&Вакуленко Алексей Владимирович &&&&&&&&\\
3\,&Грязнова Ольга Николаевна      &16.05&16.05&16.05& --- &16.05&16.05&&\\
4\,&Гуськов Илья Викторович        &&&&&&&&\\
5\,&Заруба Роман Викторович        &&&&&&&&\\
\midrule
6\,&Клейменов Кирилл               &14.05&16.05&14.05& --- &17.05&19.05&&\\
7\,&Куанышбек Канат                &&&&&&&&\\
8\,&Ломский Илья Александрович     &16.0&18.05&16.05& --- &&&&\\
9\,&Михайленко Владислав           &&&&&&&&\\
10\,&Наумов Игорь Владимирович     &17.05&18.05&&&&&&\\
\midrule
11\,&Нестерова Анастасия Глебовна  &14.05&17.05& & &19.05&19.05&21.05&21.05\\
12\,&Осокин Иван Сергеевич         &&&&&&&&\\
13\,&Плахотников Олег Игоревич     &14.05&&&&&&&\\
14\,&Таран Семён Юрьевич           &16.05&16.05&&&&&&\\
15\,&Шейнкер Алексей               &15.05& --- &15.05&&&&&\\

\bottomrule
\end{tabular}

\newpage
\subsection*{лаб3}

Клейменов -- имена CIR файлов с группы 7492, нет pdf-тегов для индексации (19.05 исправдлено)

Нестерова -- все есть, создала видео-комментарии, как пользоваться шаблоном

\newpage
\subsection*{лаб2}
Клейменов -- Опытным путём установлено ... а где измерения? 
 "Исследуем зависимость напряжения смещения усилителя от величины резисторов R1 и R2 Напряжение смещения"
"Рисунок 3 – Электрическая схема инструментального усилителя для определения напряжения смещения"
Нет файле с расширением CIR, который показывал бы измерение напряжение смещения.
Каким образом исследовалось напряжение смещения?
19.05 Измеряемая величина VM1 где таблица зависимости VM1 от R1 и зависимости VM1 от R2 ?

%На рисунке 3 измеряемая величина VM1,  

Ломский -- каким образом исследовалась зависимость напряжения смещения усилителя от величины резисторов R1 и R2?
 По каким данным  строились графики на рис 2.4 и 2.5?

Шейнкер -- нет исходных кодов, в свойствах pdf-файла автор Милена Т, усиление на ОУ3 долдно быть $K \approx 1$

\newpage
\subsection*{лаб1}
Клейменов -- нет кодов

Наумов -- все есть, нет pdf-тегjd автор, ключевые слова

Нестерова -- нет кодов, нет ЕСКД

Таран -- не в формате pdf

Шейнкер -- нет исходных кодов, в свойствах pdf-файла автор Влад Володин 


\end{document}
