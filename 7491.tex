%\documentclass[a4paper,landscape,11pt]{article}
\documentclass[a4paper,11pt]{article} 
%\usepackage{pdflscape}
\usepackage[paper=portrait,pagesize]{typearea}
%\usepackage[T1,T2A]{fontenc}
\usepackage[utf8x]{inputenc}
%\usepackage[english,ukrainian,russian]{babel} 
\usepackage[english,russian]{babel} 
\usepackage{wrapfig}
\usepackage[table,xcdraw]{xcolor}
\usepackage{booktabs}
\usepackage{pifont}
\usepackage{graphicx}
\graphicspath{ {images/} }

\usepackage{tikz}
\usepackage{siunitx}
\usepackage[american,cuteinductors,smartlabels]{circuitikz}

\usepackage{hyperref}

\usepackage{advdate}
%\usepackage{showframe} % для отладки позиции на странице
\usepackage{cancel}

%\setlength{\voffset}{-72pt} %отступ сверху - чтобы увидеть откомментарить \usepackage{showframe}
%\setlength{\voffset}{-36pt} %landscape
%\setlength{\footskip}{-20pt} %landscape
\setlength{\topmargin}{0pt} 
%\setlength{\headheight{1pt}
\setlength{\headsep}{0pt}
%\setlength{\hoffset}{-72pt} %landscape
\setlength{\marginparwidth}{0pt}
%\setlength{\textwidth}{530pt} %landscape
%\setlength{\textheight}{808pt} %landscape


%\author{ Прокшин Артем \\
%\small ЛЭТИ\\
%\small \texttt{taybola@gmail.com}}

%\date{}
%abcdefghijklmnop


\newcommand*\OK{&\small \ding{51}$\!\!_\circ$} % начал защищать
\newcommand*\Ok{&\small \ding{51}$\!\!_\circ$} % начал защищатi
\newcommand*\ok{&{\small \ding{51}}} % присутствовал
\newcommand*\oK{&{\small \ding{51}?}} % присутствовал?
\newcommand*\no{&{\small }} % отсутствовал
\newcommand*\D{\tiny\ding{48}} % защита, defend
\newcommand*\da{&{\small\ding{48}$\!\!_1$}} % защита, defend
\newcommand*\dab{&{\small\ding{48}$\!\!^1_2$}} % защита, defend
\newcommand*\ab{&{\small\ding{48}$\!\!^1_2$}} % защита, defend
\newcommand*\ad{&{\small${}^1\!\!$\ding{48}$\!\!_4$}} % защита, defend
%\newcommand*\ab{&{\small\ding{48}$\!\!^1_2$}} % защита, defend
\newcommand*\bc{&{\small\ding{48}$\!\!^2_3$}} % защита, defend
\newcommand*\dabc{&{\small\ding{48}$\!\!^1_{23}$}} % защита, defend
\newcommand*\dabcd{&{\small\ding{48}$\!\!^{12}_{34}$}} % защита, defend
\newcommand*\ac{&{\small\ding{48}$\!\!^1_{23}$}} % защита, defend
\newcommand*\db{&{\small\ding{48}$\!\!_2$}} % защита, defend
\newcommand*\dc{&{\small\ding{48}$\!\!_3$}} % защита, defend
\newcommand*\dd{&{\small\ding{48}$\!\!_4$}} % защита, defend
\newcommand*\bd{&{\small${}^2\!\!$\ding{48}$\!\!^3_{4}$}} % защита, defend
\newcommand*\de{&{\small\ding{48}$\!\!_5$}} % защита, defend
\newcommand*\dE{&{\small${}^4\!\!\!$\ding{48}$\!\!_5$}} % защита, defend
\newcommand*\cd{&{\small\ding{48}$\!\!^3_4$}} % защита, defend
\newcommand*\dg{&{\small\ding{48}$\!\!_6$}} % защита, defend
\newcommand*\fg{&{\small${}^6\!\!$\ding{48}$\!\!_7$}} % защита, defend
\newcommand*\dH{&{\small\ding{48}$\!\!_8$}} % защита, defend
\newcommand*\gh{&{\small\ding{48}$\!\!^7_8$}} % защита, defend
\newcommand*\fh{&{\small\ding{48}$\!\!^7_{89}$}} % защита, defend 
\newcommand*\ce{&{\small${}^3\!\!$\ding{48}$\!\!_5$}} % защита, defend
\newcommand*\ef{&{\small${}^5\!\!$\ding{48}$\!\!_6$}} % защита, defend
%\newcommand*\dh{&{\small\ding{48}$\!\!_8$}} % защита, defend
\newcommand*\di{&{\small\ding{48}$\!\!_9$}} % защита, defend
\newcommand*\cdef{&{\small ${}^2_4\!\!$\ding{48}$\!\!^{3}_{5}$}} % защита, defend
\newcommand*\cde{&{\small ${}^2\!\!$\ding{48}$\!\!^{3}_{5}$}} % защита, defend
\newcommand*\efg{&{\small ${}^5\!\!$\ding{48}$\!\!^{6}_{7}$}} % защита, defend
\newcommand*\befgh{&{\small ${}_2^5\!\!$\ding{48}$\!\!^{6}_{78}$}} % защита, defend
\newcommand*\Dh{&{\small${}^4\!\!$\ding{48}$\!\!_8$}} % защита, defend
\newcommand*\cfg{&{\small ${}^3\!\!$\ding{48}$\!\!^{6}_{7}$}} % защита, defend
\newcommand*\fgh{&{\small ${}^6\!\!$\ding{48}$\!\!^{7}_{8}$}} % защита, defend
\newcommand*\bce{&{\small ${}^2\!\!$\ding{48}$\!\!^{3}_{5}$}} % защита, defend
\newcommand*\dO{&{\small\ding{48}$\!\!_{15}$}}
\newcommand*\Skip{\noindent\rule{0.3cm}{0.9pt}}


\begin{document}
%\thispagestyle{empty}
% or
\pagenumbering{gobble}
%\AdvanceDate[-1] % печатаю в субботу а нужна пятница
\begin{center}\today\end{center}
\vspace*{1\baselineskip} %landscape

%\begin{table} \centering 
%\hspace{-6cm} % landscape
\hspace{-2cm} % portrait
\newcommand*{\CS}{9pt} % ширина колонки
\begin{tabular}{p{7pt}|l|p{\CS}|p{\CS}|p{\CS}|p{\CS}|p{\CS}|p{\CS}|p{\CS}|p{\CS}|p{\CS}}
%\multicolumn{16}{c}{График выполнения лабораторных работ студентами 8871 группы} \\ 
\multicolumn{11}{c}{Ведомость посещения занятий по датчикам студентами 7491 группы} \\
\toprule 
&&&&&&&&&&\\
&&&&&&&&&&\\
&&&&&&&&&&\\
&&&&&&&&&&\\
&&&&&&&&&&\\
&&&&&&&&&&\\
&&\rotatebox{90}{\rlap{\small 27 февраля ( ОУ )}}
&\rotatebox{90}{\rlap{\small 12 марта (инстр.У)}}
&\rotatebox{90}{\rlap{\small 26 марта }}
&\rotatebox{90}{\rlap{\small 9 апреля }}
&\rotatebox{90}{\rlap{\small 23 апреля }}
&\rotatebox{90}{\rlap{\small 7 мая}}
&\rotatebox{90}{\rlap{\small 21 мая }}
&\rotatebox{90}{\rlap{\small }}
&\rotatebox{90}{\rlap{\small }}
\\
\midrule
1\,&Аникин Владислав                \ok\ok\ok\ok\ok\ok&&&\\ 
2\,&Бочаров Константин Михайлович   \ok\ok\ok\ok\ok\ok&&&\\
3\,&Булычёв Валерий                 \ok\ok\no\no\no\ok&&&\\
4\,&Димов Вячеслав Викторович       \ok\no\ok\no\no\no&&&\\
5\,&Зиганшина Лилия Альфредовна     \ok\no\ok\ok\ok\ok&&&\\
\midrule
6\,&Илатовская Екатерина Вадимовна  \ok\ok\ok\ok\ok\ok&&&\\
7\,&Исакаев Ментимир                \ok\ok\ok\ok\ok\ok&&&\\
8\,&Кайгородов Дмитрий Евгеньевич   \ok\ok\ok\ok\ok\ok&&&\\
9\,&Каряева Маргарита Игоревна      \no\ok\ok\ok\ok\ok&&&\\ 
10\,&Ковалев Владимир Владимирович  \ok\ok\no\ok\ok\ok&&&\\
\midrule
11\,&Кожевников Павел Сергеевич     \ok\ok\ok\ok\ok\ok&&&\\
12\,&Кононович Михаил Александрович \ok\no\no\no\no\no&&&\\
13\,&Лазурко Андрей Владимирович    \ok\no\no\no\no\no&&&\\
14\,&Лысенко Максим Викторович      \ok\ok\no\no\ok\ok&&&\\
15\,&Мачеев Евгений Михайлович      \ok\ok\no\ok\ok\ok&&&\\
\midrule
16\,&Нурмухаметов Тимур Алмазович   \ok\ok\ok\ok\ok\ok&&&\\
17\,&Одынец Иван Михайлович         \ok\ok\no\ok\ok\ok&&&\\
18\,&Прохоров Виталий Андреевич     \ok\ok\ok\ok\ok\ok&&&\\ 
19\,&Пырков Роман Владиславович     \ok\ok\ok\ok\ok\ok&&&\\
20\,&Сизова Екатерина Сергеевна     \ok\ok\ok\ok\ok\ok&&&\\
\midrule
21\,&Силинский Алексей Николаевич   \no\ok\no\no\no\no&&&\\
22\,&Тупикова Милена                \ok\ok\ok\ok\ok\ok&&&\\
23\,&Федоркова Анастасия Олеговна   \ok\ok\ok\ok\ok\ok&&&\\
24\,&Червоная Вероника              \ok\ok\ok\ok\ok\ok&&&\\
25\,&Чжэн Сичан                     \no\ok\no\no\no\no&&&\\ 
\bottomrule
\end{tabular} 

\newpage
\KOMAoptions{paper=landscape,pagesize}
\recalctypearea
%\begin{landscape}
\hspace{-6.1cm} %landscape
\begin{tabular}{l|llccccccccccccc}
\multicolumn{10}{c}{выполнение лабораторнах работ, 7491 группа} \\
\toprule
&&Л1&Зач1& Л2&Зач2& Л3&Зач3& Л4&Зач4&  Л5&Зач5& Л6&Зач6&\\
\midrule
1\,&Аникин Владислав                &13.03&13.03 &30.03&30.03& 2.05& 7.05&21.04&21.04& 4.05& 7.05 & &\\
2\,&Бочаров Константин Михайлович   &13.03&      &     &     & 9.04& 9.04&     &     & 7.05& 7.05\\
3\,&Булычёв Валерий                 &     &      &     &     &     &     &     &\\
4\,&Димов Вячеслав Викторович       &     &13.03 &     &     &     &     &     &\\
5\,&Зиганшина Лилия Альфредовна     &23.03&23.03 &     &     & 9.04& 9.04&     &\\
\midrule
6\,&Илатовская Екатерина Вадимовна  &13.03&13.03 &23.06& 8.05& 7.05& 8.05&22.04& 8.05& 7.05& 7.05\\
7\,&Исакаев Ментимир                &13.03&13.03 &     &     &22.04& 8.05&     &     & 7.05& 7.05\\
8\,&Кайгородов Дмитрий Евгеньевич   &13.03&13.03 & 6.04& --  &31.03& --  &23.04& 7.05& 7.05& 7.05\\
9\,&Каряева Маргарита Игоревна      &19.03&19.03 &     &     & 9.04& 9.04&23.04& 8.05& 7.05& 7.05\\
10\,&Ковалев Владимир Владимирович  &13.03& ---  &     &     & 9.04& 9.04&23.04& --- & &\\
\midrule
11\,&Кожевников Павел Сергеевич     &13.03&13.03 & 9.04& 9.04& 9.04& 9.04&     &     & 5.05& 7.05\\
12\,&Кононович Михаил Александрович &     &      &     &     &     &     &     &\\
13\,&Лазурко Андрей Владимирович    &     &      &     &     &     &     &     &\\
14\,&Лысенко Максим Викторович      &13.03&13.03 &     &     &24.04&24.04& 2.05& 7.05\\
15\,&Мачеев Евгений Михайлович      &     &      &     &     &     &     &     &     & 6.05& 7.05\\
\midrule
16\,&Нурмухаметов Тимур Алмазович   &26.03&26.03 &10.04& 8.05&10.04& 8.05& 8.05& 8.05& &\\
17\,&Одынец Иван Михайлович         &30.03&30.03 &     &     &     &     &     &\\
18\,&Прохоров Виталий Андреевич     &13.03&13.03 & 7.05&     & 9.04& 9.04&     &     & 8.05& 8.05\\
19\,&Пырков Роман Владиславович     & 9.04& 8.05 & 9/04& 8.05&23.04&23.04&&\\
20\,&Сизова Екатерина Сергеевна     &13.03&13.03 &     &     & 9.04& 9.04&22.04&22.04& 7.05& 7.05\\
\midrule
21\,&Силинский Алексей Николаевич   &     &      &     &     &     &     &     &\\
22\,&Тупикова Милена                &     &      &     &     & 3.04& 9.04&     &     & 7.05& 7.05\\
23\,&Федоркова Анастасия Олеговна   &30.03&30.03 &     &     & 9.04& 9.04&     &     & 7.05&7.05\\
24\,&Червоная Вероника              &13.03&13.03 &30.03&30.03& 9.04& 9.04&     &\\
25\,&Чжэн Сичан                     &     &      &     &     &     &     &\\
\bottomrule
\end{tabular}
%	\end{landscape}

\newpage
\KOMAoptions{paper=portrait,pagesize}
\recalctypearea
\subsection*{лаб5}
Аникин -- добавлен ФНЧ, после оьсуждения решили убрать.

Бочаров -- все есть

Исакаев -- все есть, схема на рис 1 съехала и не видно входа схемы, нет кодов

Кайгородов -- все есть, шифр в рамке, договаривались,должен быть 749108\_05, нет кода

Каряева -- все есть

Кожевников -- все есть

Мачеев -- выход для 5й работы должен быть с 1го ОУ схемы для 4й работы, а не со второго 

\newpage
\subsection*{лаб4}
Аникин -- всё есть.

Буычев -- не понял: частата среза 8кГц или 23кГц? определитесь!

Ковалев -- напряжение на входе микроконтроллера 300 вольт?

Лысенко -- все есть, частота fc соответсвует варианту, на АЦП занят весь диапазон

Нурмухаметов -- все есть, отмечу оформление

Сизова -- все есть

Федоркова -- все есть

\newpage
\subsection*{лаб.3}
Аникин -- реальные ОУ LM358

Зиганшина -- сигналынизкойчастотыиобеспечиваетзатуханиевысокочастотныхсигналов. я верю, что за вас постарался word.  отчет выглядит хорошо.

Илатовская -- конвертируйте в pdf иначе вашу работу мне проверить будет проблематично.

Исакиев - в графиках АЧХ в значениях коэф усиления(безразмерного) появились отрицатеьные величины.

Кайгородов -- графики без подписей осей, ни названия оси ни единицы измерения. 

Каряева -- всё есть. в шабоне поменяйте фамилию Кайгородов

Ковалев -- похоже напутано с формулой f c = (e 3 40 / +(10−1)∗ 2 15 / + 10) ∗ 30 e 3 40 / +10 = 29,8113 Гц. Честно говоря я не понял как перевести написанное в такую строчку в формулу и ответ, 
возможно, неправильный.
           Однако, вывод по работе хорош. нет исходников схем.

Кожевников -- все есть.	  

Лысенко -- задержал работу, в выводе я не совсем понял про единицу при не/инвертирующем, нет исходников схем, дослал исходники схем

Нурмухаметов -- все есть

Прохоров -- ЕСКД докладывает об 11 листах, а в работе 12 листов, лист за 8 оставлен пустым. Хорошим тоном было бы оставить пустой лист после титульника (для черновика), но подписать при этом 
\href{https://en.wikipedia.org/wiki/Intentionally_blank_page}{"This page [is] intentionally left blank."}. Но, к сожалению, word не знает таких премудростей.

Пырков -- задержал работу, в шаблоне поменяйте Ивана Одынца. Экселевский график на рис 9 забрался ниже 0 (такого не бывает), а на рис 6 взбрыкнул вверх. такого тоже не бывает.
нет исходников схем.


Сизова -- не точечных значений, а точных значений.

Тупикова  -- все есть

Федоркова -- все есть

Червоная -- нет исходников схем



\newpage
\subsection*{лаб.2}
Аникин,   желательно выбрать $R_3= R_4 = R_5 = R_7 = 1k$, а коэф усиления выбирать с помощью $R_1, R_2$

Кайгородов -- стр 6, коэф подавдления синфазных помех  $16,2/ 8,14 \ne 1990$

Кожевников -- коэф. подавления синфазных помех есть, и все остальное тоже

Прохоров  -- уже было замечание в группе: желательно выбрать $R_3= R_4 = R_5 = R_7 = 1k$, а коэф усиления выбирать с помощью $R_1, R_2$

Пырков -- коэф подавления синф помех в норме.

Червоная -- сменила ОУ, меняя сопротивления $R_1$ и $R_2$ напряжения смещения остается неизменным.
            коэф. подавления синфазного сигнала определен неправильно. Из рис 3,4, $K_\textcyrillic{дифф} \approx \frac{2\cdot 44v }{37mv} = 2*1180 = 2360$,
	    нет АЧХ

\newpage
\subsection*{лаб.1}
Аникин, OPA277  

Димов, OPA277, почему-то 14 вариант, вместо 4

Сизова - LM324, 

Каряева: Передаточная характеристика ОУ при инвертирующем включении не соответствует схеме, приведенной выше по тексту, тоже для неинвертирюющего вскобчения,
нет таблицы, по которой строились данные

Кожевников OPA277

Ковалев -- OPA277, дифференциальное равно $\infty$, но получил какие-то значения

Исакаев   -- OPA277

Лысенко -- LM318

Нурмухаметов Тимур -- отличное оформление работы!

Пырков -- LM318

Федоркова -- желательно указать какой именно коэф усиления: ПО НАПРЯЖЕНИЮ

Червоная -- LM324



\end{document}
