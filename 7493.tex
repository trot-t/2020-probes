%\documentclass[a4paper,landscape,11pt]{article}
\documentclass[a4paper,11pt]{article} 
%\usepackage{pdflscape}
\usepackage[paper=portrait,pagesize]{typearea}
%\usepackage[T1,T2A]{fontenc}
\usepackage[utf8x]{inputenc}
%\usepackage[english,ukrainian,russian]{babel} 
\usepackage[english,russian]{babel} 
\usepackage{wrapfig}
\usepackage[table,xcdraw]{xcolor}
\usepackage{booktabs}
\usepackage{pifont}
\usepackage{graphicx}
\graphicspath{ {images/} }

\usepackage{tikz}
\usepackage{siunitx}
\usepackage[american,cuteinductors,smartlabels]{circuitikz}

\usepackage{hyperref}

\usepackage{advdate}
%\usepackage{showframe} % для отладки позиции на странице
\usepackage{cancel}

%\setlength{\voffset}{-72pt} %отступ сверху - чтобы увидеть откомментарить \usepackage{showframe}
%\setlength{\voffset}{-36pt} %landscape
%\setlength{\footskip}{-20pt} %landscape
\setlength{\topmargin}{0pt} 
%\setlength{\headheight{1pt}
\setlength{\headsep}{0pt}
%\setlength{\hoffset}{-72pt} %landscape
\setlength{\marginparwidth}{0pt}
%\setlength{\textwidth}{530pt} %landscape
%\setlength{\textheight}{808pt} %landscape


%\author{ Прокшин Артем \\
%\small ЛЭТИ\\
%\small \texttt{taybola@gmail.com}}

%\date{}
%abcdefghijklmnop


\newcommand*\OK{&\small \ding{51}$\!\!_\circ$} % начал защищать
\newcommand*\Ok{&\small \ding{51}$\!\!_\circ$} % начал защищатi
\newcommand*\ok{&{\small \ding{51}}} % присутствовал
\newcommand*\oK{&{\small \ding{51}?}} % присутствовал?
\newcommand*\no{&{\small }} % отсутствовал
\newcommand*\D{\tiny\ding{48}} % защита, defend
\newcommand*\da{&{\small\ding{48}$\!\!_1$}} % защита, defend
\newcommand*\dab{&{\small\ding{48}$\!\!^1_2$}} % защита, defend
\newcommand*\ab{&{\small\ding{48}$\!\!^1_2$}} % защита, defend
\newcommand*\ad{&{\small${}^1\!\!$\ding{48}$\!\!_4$}} % защита, defend
%\newcommand*\ab{&{\small\ding{48}$\!\!^1_2$}} % защита, defend
\newcommand*\bc{&{\small\ding{48}$\!\!^2_3$}} % защита, defend
\newcommand*\dabc{&{\small\ding{48}$\!\!^1_{23}$}} % защита, defend
\newcommand*\dabcd{&{\small\ding{48}$\!\!^{12}_{34}$}} % защита, defend
\newcommand*\ac{&{\small\ding{48}$\!\!^1_{23}$}} % защита, defend
\newcommand*\db{&{\small\ding{48}$\!\!_2$}} % защита, defend
\newcommand*\dc{&{\small\ding{48}$\!\!_3$}} % защита, defend
\newcommand*\dd{&{\small\ding{48}$\!\!_4$}} % защита, defend
\newcommand*\bd{&{\small${}^2\!\!$\ding{48}$\!\!^3_{4}$}} % защита, defend
\newcommand*\de{&{\small\ding{48}$\!\!_5$}} % защита, defend
\newcommand*\dE{&{\small${}^4\!\!\!$\ding{48}$\!\!_5$}} % защита, defend
\newcommand*\cd{&{\small\ding{48}$\!\!^3_4$}} % защита, defend
\newcommand*\dg{&{\small\ding{48}$\!\!_6$}} % защита, defend
\newcommand*\fg{&{\small${}^6\!\!$\ding{48}$\!\!_7$}} % защита, defend
\newcommand*\dH{&{\small\ding{48}$\!\!_8$}} % защита, defend
\newcommand*\gh{&{\small\ding{48}$\!\!^7_8$}} % защита, defend
\newcommand*\fh{&{\small\ding{48}$\!\!^7_{89}$}} % защита, defend 
\newcommand*\ce{&{\small${}^3\!\!$\ding{48}$\!\!_5$}} % защита, defend
\newcommand*\ef{&{\small${}^5\!\!$\ding{48}$\!\!_6$}} % защита, defend
%\newcommand*\dh{&{\small\ding{48}$\!\!_8$}} % защита, defend
\newcommand*\di{&{\small\ding{48}$\!\!_9$}} % защита, defend
\newcommand*\cdef{&{\small ${}^2_4\!\!$\ding{48}$\!\!^{3}_{5}$}} % защита, defend
\newcommand*\cde{&{\small ${}^2\!\!$\ding{48}$\!\!^{3}_{5}$}} % защита, defend
\newcommand*\efg{&{\small ${}^5\!\!$\ding{48}$\!\!^{6}_{7}$}} % защита, defend
\newcommand*\befgh{&{\small ${}_2^5\!\!$\ding{48}$\!\!^{6}_{78}$}} % защита, defend
\newcommand*\Dh{&{\small${}^4\!\!$\ding{48}$\!\!_8$}} % защита, defend
\newcommand*\cfg{&{\small ${}^3\!\!$\ding{48}$\!\!^{6}_{7}$}} % защита, defend
\newcommand*\fgh{&{\small ${}^6\!\!$\ding{48}$\!\!^{7}_{8}$}} % защита, defend
\newcommand*\bce{&{\small ${}^2\!\!$\ding{48}$\!\!^{3}_{5}$}} % защита, defend
\newcommand*\dO{&{\small\ding{48}$\!\!_{15}$}}
\newcommand*\Skip{\noindent\rule{0.3cm}{0.9pt}}


\begin{document}
%\thispagestyle{empty}
% or
\pagenumbering{gobble}
%\AdvanceDate[-1] % печатаю в субботу а нужна пятница
\begin{center}\today\end{center}
\vspace*{1\baselineskip} %landscape

%\begin{table} \centering 
%\hspace{-6cm} % landscape
\hspace{-2cm} % portrait
\newcommand*{\CS}{9pt} % ширина колонки
\begin{tabular}{p{7pt}|l|p{\CS}|p{\CS}|p{\CS}|p{\CS}|p{\CS}|p{\CS}|p{\CS}|p{\CS}|p{\CS}}
%\multicolumn{16}{c}{График выполнения лабораторных работ студентами 8871 группы} \\ 
\multicolumn{11}{c}{Ведомость посещения занятий по датчикам студентами 7493 группы} \\
\toprule 
&&&&&&&&&&\\
&&&&&&&&&&\\
&&&&&&&&&&\\
&&&&&&&&&&\\
&&&&&&&&&&\\
&&&&&&&&&&\\
&&\rotatebox{90}{\rlap{\small 5 марта ( ОУ )}}
&\rotatebox{90}{\rlap{\small 19 марта (инстр.У)}}
&\rotatebox{90}{\rlap{\small 2 апреля }}
&\rotatebox{90}{\rlap{\small 16 апреля }}
&\rotatebox{90}{\rlap{\small 30 апреля }}
&\rotatebox{90}{\rlap{\small 14 мая}}
&\rotatebox{90}{\rlap{\small 28 мая }}
&\rotatebox{90}{\rlap{\small }}
&\rotatebox{90}{\rlap{\small }}
\\
\midrule 
1\,&Буслаев Артём Сергеевич               \ok\ok\no\ok\ok\no\ok&&\\
2\,&Володин Владислав Сергеевич           \ok\ok\ok\ok\ok\ok\no&&\\
3\,&Горюнов Дмитрий Олегович              \ok\ok\no\no\no\ok\no&&\\
4\,&Григорян Артем Арменович              \ok\ok\no\no\no\no\no&&\\
5\,&Гуров Роман Станиславович             \no\no\no\no\no\no\no&&\\
\midrule
6\,&Дегтярёв Никита Витальевич            \no\no\no\no\no\no\no&&\\
7\,&Ерофеев Адам Юрьевич                  \ok\ok\ok\no\ok\ok\ok&&\\
8\,&Кабанов Алексей Андреевич             \no\ok\ok\ok\no\no\no&&\\
9\,&Казак Иван Валерьевич                 \no\no\no\no\no\no\ok&&\\ 
10\,&Каримов Мухаммадсодик Зохиджон Угли  \ok\no\ok\no\no\no\no&&\\
\midrule
11\,&Кириллов Владимир Сергеевич          \no\no\no\no\no\no\ok&&\\
12\,&Кисюк Кристина Владимировна          \no\ok\ok\ok\ok\ok\ok&&\\
13\,&Купрацевич Екатерина Викторовна      \ok\ok\ok\ok\ok\ok\ok&&\\
14\,&Кушнарев Никита Игоревич             \ok\ok\ok\ok\ok\no\ok&&\\
15\,&Макаревич Елена Олеговна             \no\ok\ok\ok\ok\ok\no&&\\
\midrule
16\,&Малов Алексей Сергеевич              \no\no\no\no\no\no\ok&&\\
17\,&Постаногова Елена Олеговна           \ok\ok\ok\ok\ok\ok\ok&&\\
18\,&Пугачев Леонид Андреевич             \ok\no\no\no\ok\ok\ok&&\\ 
19\,&Строганов Никита Вячеславович        \ok\ok\no\no\no\no\no&&\\
\bottomrule
\end{tabular} 

\newpage
\KOMAoptions{paper=landscape,pagesize}
\recalctypearea

%
\hspace{-6.1cm} %landscape
\begin{tabular}{l|llccccccccccccc}
\multicolumn{10}{c}{выполнение лабораторнах работ, 7493 группа} \\
\toprule
&&Л1&Л1& Л2&Л2& Л3&Л3& Л4&Л4& Л5&Л5\\ 
\midrule
1\,&Буслаев Артём Сергеевич               &19.03&19.03& 7.04& 1.05&16.04&16.04&24.05& --- &28.05&28.05\\
2\,&Володин Владислав Сергеевич           &19.03&19.03&22.04& --- &16.04&10.04&27.05&27.05\\
3\,&Горюнов Дмитрий Олегович              &--   &---  & 5.05&27.05&15.05&27.05&&\\
4\,&Григорян Артем Арменович              &&&&&&&&\\
5\,&Гуров Роман Станиславович             &&&&&&&&\\
\midrule
6\,&Дегтярёв Никита Витальевич            &23.05& --- &23.05& --- &23.05& --- &&\\
7\,&Ерофеев Адам Юрьевич                  &---&--&&&&&&\\
8\,&Кабанов Алексей Андреевич             &19.03&19.03&2.04& --- &16.04& --- &24.05& ---\\
9\,&Казак Иван Валерьевич                 &10.04&     &10.04 & ---&28.05&28.05&28.05&28.05\\
10\,&Каримов Мухаммадсодик Зохиджон Угли  &&&&&&&&\\
\midrule
11\,&Кириллов Владимир Сергеевич          &     &     &28.05& ---&&&&\\
12\,&Кисюк Кристина Владимировна          &19.03&19.03&2.04 &2.04&16.04&16.04 &30.04&26.05&6.05&26.05&29.05&29.05\\
13\,&Купрацевич Екатерина Викторовна      &19.03&19.03& 4.04&24.05&21.04&21.04&15.05&15.05&26.05&26.05&28.05&28.05\\
14\,&Кушнарев Никита Игоревич             &19.03&19.03& 2.04&     &16.04&16.04&23.04&30.05\\
15\,&Макаревич Елена Олеговна             &19.03& 9.05&22.04& --  &15.05&24.05&28.05&28.05\\
\midrule
16\,&Малов Алексей Сергеевич              &26.05&27.05&26.05&27.05&26.05&27.05&&&29.05&29.05\\
17\,&Постаногова Елена Олеговна           &19.03&19.03&2.04&2.04&16.04&16.04&14.05&22.05&23.05&27.05&28.05&28.05\\
18\,&Пугачев Леонид Андреевич             &&&&&&&&\\
19\,&Строганов Никита Вячеславович        &--&---&18.05& --- &18.05& --- &18.05& ---\\

\bottomrule
\end{tabular}

\newpage
\KOMAoptions{paper=portrait,pagesize}
\recalctypearea

\subsection*{Лаб 6}
Кисюк -- измерение температуры с нестабильным питанием от выпрямителя без фильтра будет давать неверный результат

Купрацевич -- все есть

\subsection*{Лаб 5}
Буслаев -- все есть.

Кисюк -- все есть, даже формат PDF/A

Купрацевич -- все есть
\subsection*{Лаб 4}

Володин --  как автор в pdf-свойствах указано vikitan.krav.first@gmail.com ()

Буслаев -- нет графиков напряжения на выходе схемы (на входе микроконтроллера). в свойствах pdf-файла стоит Антон Исаков,
нет ни темы, ни даты, ни ключевых слов.

Кабанов -- нет исходных кодов, какие сигналы на выходе схемы, неясно по отчету.  в свойствах pdf-файла минимум информации,
только имя, нет темы, ключевых слов.

Казак -- все есть

Кисюк -- на рис 3 сигнал на входе АЦП около 230 в. но АЦП сгорит если на него подать сигнал выше 3.3в.  Определим действующее напряжение сети:
U=(110+10·12)=230 В  -- здесь покажите параметры, которые вы задаете для VG1

Кушнарев -- напряжение на выходе схемы сожжет микроконтроллер (мах 3.3в), в свойствах pdf-файла минимум информации,
только ФИО, нет темы, ключевых слов.

\newpage
\subsection*{Лаб 3}

Володин -- нет кодов, все есть

Горюнов -- без рамок ЕСКД, нет исходных кодов

Дягтерев --  В свойствах pdf-файла нет автора,темы,... нет исходных кодов.

Кабанов -- нет кодов tina

Казак -- все есть

Кисюк -- все есть

Купрацевич -- в таблице 1 вы видите что U не менеятся (а это то что мы измеряем), не кажутся вам ваши выводы странными? 

\newpage
\subsection*{Лаб 2}

Бусаев -- все есть.

Володин -- нет кодов, каким образом был найден коэффициент подавления синфазных помех

Горюнов -- коэф. подавления синфазных помехх катострофически мал, Исследоватьзависимость напряжения смещения усилителя от величины резисторов R1 и R2
-- из отчета неясно откуда взялись цифры в таблице, в свойствах pdf-файла стоит автором Кмения, кто такая? нет исходных кодов. 
(15 мая Коэф подавления синфазных помех $141*10^3$, скорее всего не мал)

Дягтерев --  В свойствах pdf-файла нет автора,темы,... нет исходных кодов. Где опытные данные для этого утверждения?: Опытным путём установлено, что с увеличением R1 напряжение смещения увеличивается, а с увеличением R2 уменьшается

Казак -- коэффициент подавления синфазного сигнала неверен.
обсуждение было в канале телеграм

Кириллов -- усиление должно быть достигнуто R2 и R1, усиление сопротивлением R3, R4 приводит к преждевременному насыщению и неверным результатам.


Купрацевич -- коэф. подавления синфазных помех был неверенo
напряжение смещения...
измеряемая величина Uвых и она не меняется, это видим экспериментально,
но это не напряжение смещения! для Eсм у вас есть графа и оно в ней меняется, но этим цифрам я не верю потому что не меняется Uвых.
Что-то не так.
Потому что для графы Eсм вы используете формулу которой почему-то верите (почему? кстати)

на каждом ОУ есть напряжение смещения на входе(в нашем случае при моделировании оно одинаковое для всех трех). мы наблюдаем за выходом. от R1 и R2 не зависит. может быть зависит от R3 и R4? значит должно быть преобразование или формула.
Какая именно? Если нет теоретической, тогда хотя бы экспериментальной зависимости от R3,R4

Опытным путём установлено, что с увеличением R1 или R2 напряже-ние смещения практически не изменяется -- а в таблице меняется


Макаревич -- неправильный коэф подавления синфазных помех.

\newpage
Лаб1 

Буслаев  -- очень кратко, не найден коэф подавления синфазных помех

Володин -- LM324  желательно $R_1=R_2$ и $R_3=R_4$, не найден коэф подавления синфазных помех

Горюнов  -- нет ЕСКД, LM318, кроме оформления все хорошо

Дягтерев -- сконвертировать в pdf - у меня линукс и вордом я не пользуюсь, и также исходные коды по всем заданиям в форматах tina и экспорт в pscpice. Формат файлов 749306\_01XX.CIR 
и 749306\_01XX.TSC - где XX - номер задания в лабораторной. Для остальных лабораторных такие же правила. 

Ерофеев -- все есть, есть тэги pdf-файла для индексации.

Кабанов  не найден коэф подавления синфазных помех

Казак -- нет исходных кодоа

Купрацевич -- все есть

Кисюк LM307

Кушнарев LM318

Постаногова LM318 желательно $R_1=R_2$ и $R_3=R_4$, лучше использовать симметричные значения $R_i$ : $R_1=7k$, $R_2=7k$. 

Строганов нет ЕСКД, LM318, кроме оформления все хорошо
\end{document}
